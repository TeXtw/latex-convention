\subsection{二種冒號(colon)}

\begin{Wrong}
\begin{verbatim}
\[\{\, x \colon x \notin x \,\}\]
\[f : x \to x^2\]
\end{verbatim}
\[\{\, x \colon x \notin x \,\}\]
\[f : x \to x^2\]
\end{Wrong}

\begin{Right}
\begin{verbatim}
\[\{\, x : x \notin x \,\}\]
\[f \colon x \to x^2\]
\end{verbatim}
\[\{\, x : x \notin x \,\}\]
\[f \colon x \to x^2\]
\end{Right}

二種(英文)冒號 \verb|:| 和 \verb|\colon| 表現出來,在形狀上雖然相同,但是置放位置不同。
通常 \verb|:| 是用在集合描述(關係運算符號),而 \verb|\colon| 是當成標點符號,
常用在映射表示\footnote{請參考\href{https://tex.stackexchange.com/questions/37789/using-colon-or-in-formulas}{\sf StackExchange}的討論。}。
另外,比例通常用 \verb|:|,例如$x:y:z = 3:4:5$。

\marginpar{\back}
