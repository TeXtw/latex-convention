\subsection{單位前需要小空白}
\label{sub:unit}

\begin{Wrong}
\begin{verbatim}
台灣南北的長度大約 394km,東西上大闊度大約 144km。
\end{verbatim}

台灣南北的長度大約 394km,東西上大闊度大約 144km。
\end{Wrong}

\begin{Right}
\begin{verbatim}
台灣南北的長度大約 394\,km,東西上大闊度大約 144\,km。
\end{verbatim}

台灣南北的長度大約 394\,km,東西上大闊度大約 144\,km。
\end{Right}

單位前需要一個小空白(插入\verb|\,|)。可以參考\href{https://github.com/josephwright/siunitx}{\sf siunitx}套件的例子。
如果不想傷這個腦筋,可以引用{\sf siunitx}套件,依照它的使用方法來表現數字及單位,
這樣就可以全文一致,不必一個一個去手動修正。

有規則就會有例外,非字母的符號(準)單位,例如溫度(24\textcelsius)、百分號(10\%),數字和符號間是不留空白的。a.m./p.m.\/這類表示上下午的也不留空白\footnote{正式文件一般主張要留空白,2:45\,p.m.,而且時區要用小括號括住,小括號前要留空白,2:45\,p.m.\,(EST)。}。

{\sf SI}(Système International d'Unités)對此有異議,認為只要是單位就得留白(\SI{24}{\degreeCelsius}、\SI{10}{\percent}),例外是單純的角度符號(\ang{30;8;22})就需要緊密。另外 \%\ 並非是{\sf SI}中認定的單位(但{\sf siunitx}中有定義百分號)。
在一般的寫作文件,對溫度度數及百分號不少人還是持不要空白的風格\footnote{Chicago style主張不要空白,APA及ACS style主張要空白。請參考:\url{https://blogs.millersville.edu/bduncan/numbers/}。}。

好像有點亂,所以寫正式文件時,有style manual的話,還是要詳細讀一讀,upstream的要求比你的喜好重要。如果不知何去何從,最簡單的方式就是依標準來。
