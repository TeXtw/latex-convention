\subsection{單位前需要小空白}
\label{sub:unit}

\begin{Wrong}
\begin{verbatim}
台灣南北的長度大約 394km,東西上大闊度大約 144km。
\end{verbatim}

台灣南北的長度大約 394km,東西上大闊度大約 144km。
\end{Wrong}

\begin{Right}
\begin{verbatim}
台灣南北的長度大約 394\,km,東西上大闊度大約 144\,km。
\end{verbatim}

台灣南北的長度大約 394\,km,東西上大闊度大約 144\,km。
\end{Right}

單位前需要一個小空白(插入\char`\\,)。可以參考{\sf siunitx}套件的例子。
如果不想傷這個腦筋,可以引用{\sf siunitx}套件,依照它的使用方法來表現數字及單位,
這樣就可以全文一致,不必一個一個去手動修正。

% 有規則就會有例外,非字母的符號(準)單位,例如溫度(℃)、百分號(\%),數字和符
% 號間是不留空白的。am/pm這類表示上下午的也不留空白。
% SI 對此有異議,認為只要是單位就得留白,例外是單純的角度符號(30°)
% 及弧分鐘(22°)及 8" 等就需要緊密。另外 % 並非是 SI 中的單位。
% 2° 3'  4"
% 但一般的寫作文件,對度數及百分號還是持不要空白的風格。
