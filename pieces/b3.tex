\subsection{改變字型大小要用 \char`\\\texttt{par} 來調整行距}

\def\mytext{%
《說文解字》書名。東漢許慎撰,三十卷,為我國第一部有系統分析字形及考究字源的字書。按文字形體及偏旁構造分列五百四十部,首創部首編排法?}

\begin{Wrong}
\begin{verbatim}
\begin{document}
\mytext

{\footnotesize \mytext}
\end{document}
\end{verbatim}
\mytext

{\footnotesize \mytext}
\end{Wrong}

\begin{Right}
\begin{verbatim}
\begin{document}
\mytext

{\footnotesize \mytext\par}
\end{document}
\end{verbatim}
\mytext

{\footnotesize \mytext\par}
\end{Right}
那個 \char`\\\texttt{mytext}是事先定義好的一段文字。在改變字型大小時要注意它的行距,它會依原先的行距來排版,要校正這個問題,要在 group 的 \char`\}\ 之前先換成下一個段落。

不做這樣的調整的話,例子裡頭的小字(footnotesize)的段落,它的行距太大,因為是依原先normalsize的行距來排版(\TeX 是依段落來斷行的,在此之前一切資訊未定)。小字應依小字的比例來縮小行距,同理改變成大字時,也應依大字的行距來照比例調大。請參考\href{https://tex.stackexchange.com/questions/444039/why-do-i-have-to-use-par-if-i-change-font-size-withing-a-group-scope}{\textsf{StackExchange}}上的討論。

\marginpar{\back}
