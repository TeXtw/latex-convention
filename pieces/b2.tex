\subsection{一些特殊字元不能直接鍵入}
\label{sub:special}

\begin{Wrong}
\begin{verbatim}
TeX 裡頭有一些特殊字元是無法直接鍵入的,例如倒斜線 \,那是指令的引頭字元,直接鍵入編譯也不會過。
\end{verbatim}
\end{Wrong}

\begin{Right}
\begin{verbatim}
那麼這類字元要如何鍵入呢?可以使用 \ 去 escape 它,但唯獨這個倒斜線不行,要用 \textbackslash 來鍵入,或者進入數學模式 $\backslash$,這些指令都滿長的。另有簡單的方式,就是直接取出字元來 \char`\\ 就可以了(那個 ` 是左單引號)。
\end{verbatim}
\end{Right}

以下說明各種特殊符號的鍵入方式。至於取巧用 \verb|\char`\\| 的方式,在章節標題時
最好不用,因為 pdf bookmarks 的顯示會不正常(無法正確轉換為ascii或UTF-16BE)。

%{\small
\begin{center}
\setstretch{1.0}
\begin{tabular}{llll}
符號 & 作用 & 文稿上使用 & \LaTeX\ 的替代指令 \\
\hline
\textbackslash & 下排版命令 & \verb|$\backslash$| & \verb|\textbackslash|\\
\%             & 註解       & \verb|\%|           & NA \\
\#             & 定義巨集   & \verb|\#|           & NA \\
\~{}           & 產生一個空白   & \verb|\~{}|     & \verb|\textasciitilde| \\
\$             & 進入(離開)數學模式 & \verb|\$| & \verb|\textdollar| \\
\_{}           & 數學模式中產生下標字 & \verb|\_{}| & \verb|\textunderscore| \\
\^{}           & 數學模式中產生上標字 & \verb|\^{}| & \verb|\textasciicircum| \\
\{             & 標示命令的作用範圍   & \verb|\{| & \verb|\textbraceleft|\\
\}             & 標示命令的作用範圍   & \verb|\}| & \verb|\textbraceright|\\
\textless      & 數學模式中的小於符號 & \verb|$<$| & \verb|\textless| \\
\textgreater & 數學模式中的大於符號   & \verb|$>$| & \verb|\textgreater| \\
\textbar     & OT1,數學模式中才能正確顯示 & \verb+$|$+ & \verb|\textbar| \\
\&           & 表格中的分隔符號   & \verb|\&| & NA
\end{tabular}
\end{center}
%\par}  % \par fixes the line spacing.
