\subsection{重複載入的套件}

\begin{Wrong}
\begin{verbatim}
\usepackage{hyperref}
\usepackage{url}
\end{verbatim}

\end{Wrong}

\begin{Right}
\begin{verbatim}
\usepackage{hyperref}
\end{verbatim}

\end{Right}

在\LaTeX 的使用上,引用套件(package)是避免不了,但\LaTeX 的套件,超過
四千個,這麼多的套件,其中難免會有衝突。很不幸的,並沒有很好的工具來預知
哪些套件會衝突,只能靠使用過的人的經驗及自行使用時的發現。

一些套件會預設載入其他套件,這樣這些預設會載入的套件就無需重複載入了。不
過,也很不幸的,並沒有一個完美的工具預知某套件會預設載入哪些其他的套件,
除非你打開這個套件的原始碼,去看看預設載入了什麼套件。或者加入一行 \verb|\listfiles| 於
其他套件載入之前,然後編譯後開啟 \verb|*.log| 檔,找到 \verb|*File List*| 的地方,
會列出所使用的套件及其版本\footnote{有一個很dirty的小程式{\tt ltxpkg},可以在\url{https://github.com/qtnez/luatexja-template/tree/main/tools}找到。}。

這個例子裡頭,{\sf hyperref}套件,預設就是會載入{\sf url}套件,因此無需
重複載入。那麼如果想傳參數給{\sf url}時怎麼辦?這時可以在{\sf hyperef}之前載入
{\sf url}並指定參數。

\begin{Code}
\setstretch{1.0}
\begin{verbatim}
\documentclass{article}
\usepackage[hyphens]{url}
\usepackage{hyperref}
\end{verbatim}
\end{Code}

\marginpar{\back}
