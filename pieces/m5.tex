\subsection{學模式中改變字體大小}

\begin{Wrong}
\begin{verbatim}
$\huge f(x,y)=\frac{3(x+y)y}{(2xy-7)}$
\end{verbatim}
$\huge f(x,y)=\frac{3(x+y)y}{(2xy-7)}$
\end{Wrong}

\begin{Right}
\begin{verbatim}
\huge $f(x,y)=\frac{3(x+y)y}{(2xy-7)}$
\end{verbatim}
\huge $f(x,y)=\frac{3(x+y)y}{(2xy-7)}$
\end{Right}

在數學模式中,不能使用正常文字模式中改變字體大小的指令,這不僅不能改變字體大小,
而且編譯時還會出現錯誤訊息:

\begin{Code}
\begin{verbatim}
LaTeX Font Warning: Command \huge invalid in math mode on input line 38.
\end{verbatim}
\end{Code}

要改變數學模式中的字體大小,可以使用數學模式下專用的指令,或者將改變字體大小的指令移出
數學模式外。以下是能用於數學模式的指令:

\vspace{.5\baselineskip}
\begin{center}
\setstretch{1.2}
\begin{tabular}{ll}
\hline
指令 & 作用 \\
\hline
\verb|\displaystyle|      & 正常大小字體的展式數式 \\
\verb|\scriptstyle|       & 縮小成上下標字體的大小 \\
\verb|\scriptscriptstyle| & 縮小二倍上下標字體的大小 \\
\verb|\textstyle|         & 回復成正常字體大小 \\
\hline
\end{tabular}
\end{center}
\vspace{.5\baselineskip}

有引用{\sf amsmath}套件的話,可以在 \verb|\text{}| 中使用文字模式下的改變字體大小的指令:

\begin{Code}
\setstretch{1.0}
\begin{verbatim}
\usepackage{amsmath}
...
$\text{\huge $f(x,y)=\frac{3(x+y)y}{(2xy-7)}$}$
\end{verbatim}
\end{Code}

引用{\sf graphics}套件,可以在數學模式中使用 \verb|\scalebox| 來改變字體的大小。

\begin{Code}
\setstretch{1.0}
\begin{verbatim}
\usepackage{graphicx}
\newcommand*{\Scale}[2][4]{\scalebox{#1}{\ensuremath{#2}}}
...
$\Scale[1.2]{f(x,y)=\frac{3(x+y)y}{(2xy-7)}}$
\end{verbatim}
\end{Code}

\marginpar{\back}
