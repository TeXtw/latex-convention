\subsection{數學模式下使用正體字的情況}

\begin{Wrong}
\begin{verbatim}
$cos 2x=cos^2x-sin^2x$
\[lim_{n \to \infty}\sum_{i=1}^n{\frac{1}{n}}\]
\end{verbatim}
$cos 2x=cos^2x-sin^2x$
\[lim_{n \to \infty}\sum_{i=1}^n{\frac{1}{n}}\]
\end{Wrong}

\begin{Right}
\begin{verbatim}
$\cos 2x=\cos^2x-\sin^2x$
\[\lim_{n \to \infty}\sum_{i=1}^n{\frac{1}{n}}\]
\end{verbatim}
$\cos 2x=\cos^2x-\sin^2x$
\[\lim_{n \to \infty}\sum_{i=1}^n{\frac{1}{n}}\]
\end{Right}

數學模式下,為了容易區分字母所代表的意義,大原則是變數使用數學斜體。
但不是變數的情況,會有種種的慣例或標準來規範。

這裡整理一下{\sf ISO 80000-2}標準的相關說明:

\begin{itemize}
\item 變數(variables),例如$x$、$y$……等等。變動的數字(running numbers),例如$x_i$中的$i$。要用數學斜體(italic)。
\item 由於敘述上產生的函數,例如$f$、$g$,要用數學斜體。但已經是明確定義的固定函數,要用正體,例如$\sin$、$\exp$、$\ln$、$\lim$、$\log$……等等,要用正體。
\item 數學常數,例如$\text{e}=\num{2.718281828}\cdots$、$\uppi=\num{3.141592}\cdots$……等等,e 及$\uppi$要用正體。
\item 已經完整定義的運算子,例如微分符號($\text{d}x/\text{d}y$)、加減乘除以及數字,要用正體。
\end{itemize}

\marginpar{\back}

