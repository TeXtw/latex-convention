\subsection{數式環境下上不可留白}

\begin{Wrong}
\begin{verbatim}
這是一個測試。

\begin{equation*}
g(x)=\sqrt{9-x^2}
\end{equation*}
\end{verbatim}
這是一個測試。

\begin{equation*}
g(x)=\sqrt{9-x^2}
\end{equation*}
\end{Wrong}

\begin{Right}
\begin{verbatim}
這是一個測試。
\begin{equation*}
g(x)=\sqrt{9-x^2}
\end{equation*}
\end{verbatim}
這是一個測試。
\begin{equation*}
g(x)=\sqrt{9-x^2}
\end{equation*}
\end{Right}

這裡會發現留有空白行的{\tt equation}環境,上面會多空出更多的空白行出來。
要在數式環境變動上下空白,要由特定的指令來完成,不能手動的去留個空白行,
這為這個空白行被\TeX 解釋成段落。

調整數式上下的空白,可以使用特定的指令,例如:

\begin{Code}
\setstretch{1.0}
\begin{verbatim}
\abovedisplayshortskip=10pt
\belowdisplayshortskip=10pt
\begin{equation}
g(x)=\sqrt{9-x^2}
\end{equation}
\end{verbatim}
\end{Code}

同樣的情況發生在 \verb|\[...\]|,上下亦不可多出空白行。隨文數式,
因為它是要「隨文」的,因此在 \verb|\(...\)| 或是 \verb|$...$| 前後則要有個空白,
同義的 \verb|\begin{math}...\end{math}| 同樣是不能有多餘的空白行的。

\marginpar{\back}
