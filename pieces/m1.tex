\subsection{數理式子需要適當的空白}

\begin{Wrong}
\begin{verbatim}
\[L(s)=\int^b_a\sqrt{1+(f'(x))^2}dx.\]
\end{verbatim}
\[L(s)=\int^b_a\sqrt{1+(f'(x))^2}dx.\]
\end{Wrong}

\begin{Right}
\begin{verbatim}
\[L(s)=\int^b_a \! \sqrt{1+(f'(x))^2}\,\text{d}x.\]
\end{verbatim}
\[L(s)=\int^b_a \! \sqrt{1+(f'(x))^2} \, \text{d}x.\]
\end{Right}

微分符號本身是一種運算子(operator),並不是變數,在數學模式\LaTeX 需要
數學斜體的變數來和其它字母做區分,他在此的地位類似單位,之前要留一個小空
白,而且必需使用正體(upright),不能使用數學斜體(italic)。這在\href{https://saso.gov.sa/ar/mediacenter/public_multimedia/Documents/SASO-ISO-800000-2-2020-E.pdf}{ISO 80000-2}標準裡頭也是如此認定。

但是這個微分符號是否要正體是有爭議的,大體而言,數學家較多偏向用數學
斜體(包括 Knuth 本人也是使用數學斜體),物理學家較多偏向
用正體\footnote{可以參考\href{https://tex.stackexchange.com/questions/14821/whats-the-proper-way-to-typeset-a-differential-operator}{\sf StackExchange}的討論。}。不過,既然有標準出現了,大家還是盡量遵循標準比較恰當\footnote{Knuth在設計\TeX 的時候,這個標準還沒有出現。}。

使用 \verb|\text{d}| 需要{\sf amsmath}套件,否則要使用 \LaTeX 內建的 \verb|\mathrm{d}|。另外{\sf physics}套件有提供 \verb|\dd, \dv| 的方便短指令。
另有一種取巧的方式,就是使用 \verb|\mbox{d}|,被 \verb|\mbox{}| 包住的文字都會使用
正體。這幾種方式會有小差異,\verb|\mbox{}| 的方式盡量避免,請試著編譯\\
{\small \verb|$x^{\mathrm{a test}}x^{\mbox{a test}}x^{\text{a test}}x^{\textrm{a test}}$|}\\
看出來的結果就知道了。

以下列出有關空白的指令:

\vspace{.5\baselineskip}
\begin{center}
\setstretch{1.2}
\begin{tabular}{llll}
\hline
指令 & 作用 & 指令 & 作用 \\
\hline
\verb|\quad| & 空出一個 em 單位的空白 & \verb|\qquad| & 空出兩個 em 的空白 \\
\verb|\,| & 加入 1/6 quad 的空白 & \verb|\!| & 減去 1/6 quad 的空白 \\
\verb|\;| & 加入 5/18 quad 的空白 & \verb|\:| & 加入 2/9 quad 的空白\\
\hline
\end{tabular}
\end{center}
\vspace{.5\baselineskip}

這裡要注意的地方是,自從\LaTeX\ 2020-10-01發行後,一些以往只能用在數學模式的
空白指令,現在已經可以用在文字模式及數學模式了。在此之前,
\verb|\,|、\verb|\quad| 及 \verb|\qquad| 可以用在一般的文字模式及數學模式,
其他的只能使用在數學模式中。

\marginpar{\back}
