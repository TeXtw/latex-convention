\subsection{英文句點後的空白}

\begin{Wrong}
\begin{verbatim}
Please see Appendix A. We will be there soon.
\end{verbatim}
Please see Appendix A. We will be there soon.
\end{Wrong}

\begin{Right}
\begin{verbatim}
Please see Appendix A\@. We will be there soon.
\end{verbatim}
Please see Appendix A\@. We will be there soon.
\end{Right}

\LaTeX 會在英文文章的一個句子結束和另一個句子開始的中間,自動調整成較大一點的空白,
這可以增加文章的易讀性。所謂一個句子結束,例如:句點(.)、問號(?)、驚嘆號(!)及
冒號(:)。通常句子結束的最後一個英文字母是小寫的,但這個例子是一個特殊的例外,句點
之前是大寫的英文字母,\LaTeX 不會判斷成句子結尾,因此不會加大其後的空白,這時我們只好手動去告知\LaTeX 這是句子結束。

相同的情形發生在以下的段落:

\begin{Code}
\small
\setstretch{1.0}
\begin{verbatim}
Aesop lived in Ancient Greece in the 6th Century BC\@. His fables are
usually short and end with a moral.
\end{verbatim}
\end{Code}

相反的情形,句點前的字母是小寫的,但卻不是句子的結束,這時怎麼辦?這時可以在句點結束後直接加上 \verb|\ |(倒斜線後是一個空白),這樣就會抑制\LaTeX 去判斷成句子結束,例如:

\begin{Code}
\small
\begin{verbatim}
e.g.\ this is a test.
\end{verbatim}
\end{Code}

或者採用Chicago style的方式,就是在{\tt e.g.}後加一個英文逗點({\tt e.g.,}),這樣就不會被判斷成句子結束了(類似的例子還有{\tt i.e.,})。其他有使用到縮寫字的場合,例如:`Dr.'、`etc.'、`vs.'、`Fig.'、`cf.'、`Mr.'、`Mrs.',這些都不是代表句子結束,所以,要插入一個正常空白。

有人不喜歡這個規則(例如法國人),如果要中止這種判斷的話,可以在文稿的導言區加上 \verb|\frenchspacing|。

\marginpar{\back}
