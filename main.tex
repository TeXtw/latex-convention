% !TEX program = LuaLaTeX + luatex-ja
% Copyright © 2021 Guoo Jehng Lee
% CC BY 4.0 license.

\documentclass[11pt,a4paper]{article}
\usepackage{ltj-zhfonts,parskip,mdframed,luamplib,libertinus}
\usepackage[pdfstartview={FitH},pdfencoding={auto},
  pdfauthor={李果正},pdfsubject={LaTeX 正誤手冊},
  pdftitle={LaTeX 正誤手冊},pdfkeywords={LaTeX,convention},
  bookmarksopen={true}]{hyperref}

% 章節後不會留有正常空白,需要 startcode 校正。
\newmdenv[linecolor=red,frametitle=誤:,leftmargin=1em,rightmargin=1em,
  startcode=\leavevmode,topline=false,bottomline=false,linewidth=1.2pt,
  skipabove=2ex]{Wrong}
\newmdenv[linecolor=blue,frametitle=正:,leftmargin=1em,rightmargin=1em,
  topline=false,bottomline=false,linewidth=1.2pt,
  skipabove=2ex]{Right}

\title{\textsf{\textbf{\LaTeX 正誤手冊}}}
\author{李果正Guoo Jehng Lee}
\date{\zhtoday}

\begin{document}

\maketitle

排版有許多慣例,甚至最後形成標準。這份文件主要就是排除違反慣例的情況。當然所謂慣例,除非形成標準,要不然仍然是會有爭議(就算是形成了標準,也還是有些人不願意遵守),這無所謂對、錯,這份文件的「正誤」也只是針對多數人遵從的慣例而言,不是完全的非黑即白。

但是關於\TeX/\LaTeX 的語法,這就是非黑即白了,語法錯誤,嚴重的會使編譯中止,文件出不來,輕一點的是排版結果不符合預期。因此這份文件的所謂正誤,也包括了語法上的錯誤。

\section{基礎語法}

\subsection{單位前需要小空白}

%這是測試文!

\begin{Wrong}
\begin{verbatim}
台灣南北的長度大約 394km,東西上大闊度大約 144km。
\end{verbatim}
\end{Wrong}

台灣南北的長度大約 394km,東西上大闊度大約 144km。

\begin{Right}
\begin{verbatim}
台灣南北的長度大約 394\,km,東西上大闊度大約 144\,km。
\end{verbatim}
\end{Right}

台灣南北的長度大約 394\,km,東西上大闊度大約 144\,km。

單位前需要一個小空白(插入\char`\\,)。可以參考{\sf siunitx}套件的例子。
如果不想傷這個腦筋,可以引用{\sf siunitx}套件,依照它的使用方法來表現數字及單位,
這樣就可以全文一致,不必一個一個去手動修正。

\subsection{一些特殊字元不能直接鍵入}

\begin{Wrong}
\begin{verbatim}
TeX 裡頭有一些特殊字元是無法直接鍵入的,例如倒斜線 \,那是指令的引頭字元,直接鍵入編譯也不會過。
\end{verbatim}
\end{Wrong}

\begin{Right}
\begin{verbatim}
那麼這類字元要如何鍵入呢?可以使用 \ 去 escape 它,但唯獨這個倒斜線不行,要用 \textbackslash 來鍵入,或者進入數學模式 $\backslash$,這些指令都滿長的。另有簡單的方式,就是直接取出字元來 \char`\\ 就可以了(那個 ` 是左單引號)。
\end{verbatim}
\end{Right}

以下說明各種特殊符號的鍵入方式。

\begin{tabular}{llll}
符號 & 作用 & 文稿上使用 & \LaTeX\ 的替代指令 \\
\hline
\textbackslash & 下排版命令 & \verb|$\backslash$| & \verb|\textbackslash|\\
\%             & 註解       & \verb|\%|           & NA \\
\#             & 定義巨集   & \verb|\#|           & NA \\
\~{}           & 產生一個空白   & \verb|\~{}|     & \verb|\textasciitilde| \\
\$             & 進入(離開)數學模式 & \verb|\$| & \verb|\textdollar| \\
\_{}           & 數學模式中產生下標字 & \verb|\_{}| & \verb|\textunderscore| \\
\^{}           & 數學模式中產生上標字 & \verb|\^{}| & \verb|\textasciicircum| \\
\{             & 標示命令的作用範圍   & \verb|\{| & \verb|\textbraceleft|\\
\}             & 標示命令的作用範圍   & \verb|\}| & \verb|\textbraceright|\\
\textless      & 數學模式中的小於符號 & \verb|$<$| & \verb|\textless| \\
\textgreater & 數學模式中的大於符號   & \verb|$>$| & \verb|\textgreater| \\
\textbar     & OT1 編碼,數學模式中才能正確顯示 & \verb+$|$+ & \verb|\textbar| \\
\&           & 表格中的分隔符號   & \verb|\&| & NA
\end{tabular}


\section{套件使用}

\section{使用中文}

\section{數理式子}

\section{圖表處理}

\section{索引、文獻參考}

\end{document}
