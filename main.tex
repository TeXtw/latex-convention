% !TEX program = LuaLaTeX + luatex-ja
% Copyright © 2021 Guoo Jehng Lee
% CC BY 4.0 license.

\documentclass[11pt,a4paper]{article}
\usepackage{ltj-zhfonts,parskip,mdframed,listings,luamplib,libertinus}

\title{\textsf{\textbf{\LaTeX 正誤手冊}}}
\author{李果正Guoo Jehng Lee}
\date{\zhtoday}

\newmdenv[linecolor=red,frametitle=誤:,leftmargin=1em,rightmargin=1em,
          topline=false,bottomline=false,linewidth=1.2pt]{Wrong}
\newmdenv[linecolor=blue,frametitle=正:,leftmargin=1em,rightmargin=1em,
          topline=false,bottomline=false,linewidth=1.2pt]{Right}

\begin{document}
\maketitle
排版有許多慣例,甚至最後形成標準。這份文件主要就是排除違反慣例的情況。當然所謂慣例,除非形成標準,要不然仍然是會有爭議(就算是形成了標準,也還是有些人不願意遵守),這無所謂對、錯,這份文件的「正誤」也只是針對多數人遵從的慣例而言,不是完全的非黑即白。

但是關於\TeX/\LaTeX 的語法,這就是非黑即白了,語法錯誤,嚴重的會使編譯中止,文件出不來,輕一點的是排版結果不符合預期。因此這份文件的所謂正誤,也包括了語法上的錯誤。

\section{基礎語法}


\begin{Wrong}
台灣南北的長度大約 394km,東西上大闊度大約 144km。
\end{Wrong}

\begin{Right}
台灣南北的長度大約 394\,km,東西上大闊度大約 144\,km。
\end{Right}

單位前需要一個小空白(插入$\backslash$,)。可以參考{\sf siunitx}套件的例子。
如果不想傷這個腦筋,可以引用{\sf siunitx}套件,這樣對於數籽及單位的表現,
就可以全文一致。

\section{套件使用}

\section{使用中文}

\section{數理式子}

\section{圖表處理}

\section{索引、文獻參考}

\end{document}
